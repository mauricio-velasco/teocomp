
\documentclass[12pt, a4paper]{article}
\usepackage{hyperref}
\hypersetup{
  colorlinks=true,
  linkcolor=blue,
  urlcolor=cyan,
}
\urlstyle{same}
\usepackage[utf8]{inputenc}
\usepackage{amsmath}
\usepackage{amsfonts}
\usepackage{amssymb}
\usepackage{graphicx}


\newtheorem{theorem}{Teorema.}
\newtheorem{lemma}[theorem]{Lema.}
\newtheorem{corollary}[theorem]{Corolario.}
\newtheorem{definition}[theorem]{Definici\'on:}
\newtheorem{example}[theorem]{Ejemplo:}
\newtheorem{problema}[theorem]{Problema:}
\newtheorem{remark}[theorem]{Observaci\'on:}

\usepackage{graphicx}
\usepackage[spanish]{babel}
%\usetheme{default}

\newcommand{\pp}{\mathbb{P}}
\newcommand{\zz}{\mathbb{Z}}
\newcommand{\rr}{\mathbb{R}}
\newcommand{\qq}{\mathbb{Q}}
\newcommand{\RR}{\mathbb{R}}
\newcommand{\EE}{\mathbb{E}}

\usepackage{tikz, tikz-3dplot}

\definecolor{cof}{RGB}{219,144,71}
\definecolor{pur}{RGB}{186,146,162}
\definecolor{greeo}{RGB}{91,173,69}
\definecolor{greet}{RGB}{52,111,72}

\date{}

\begin{document}
\title{Pr\'actico 3 TEOCOMP: Complejidad Computacional.}
\author{Mauricio Velasco}
\maketitle{}


\begin{enumerate}
\item Demuestre formalmente que el siguiente problema de decisión esta en la clase de complejidad $P$: {\it Dado un grafo $G$ con costos en las aristas y un entero $h$, existe un árbol generador de $G$ con costo $\leq h$?}

\item Demuestre las siguientes afirmaciones sobre lenguajes $L\subseteq \{0,1\}^*$:
\begin{enumerate}
\item Si $L_1,L_2\in P$ entonces $L_1\cap L_2\in P$
\item Si $L_1 \in P$ entonces el complemento $\{0,1\}^*\setminus L_1\in P$
\item Si $L_1,L_2\in NP$ entonces $L_1\cap L_2\in NP$
\end{enumerate}

\item Sea $G$ un grafo no dirigido. Un conjunto $I\subseteq V(G)$ es independiente si $\forall a,b\in I$ tenemos que $(a,b)\not\in E(G)$. Demuestre formalmente que el siguiente problema de decisión esta en la clase de complejidad $NP$: {\it Dado un grafo $G$ y un entero $p$, existe un conjunto independiente con por lo menos $p$ v\'ertices de $G$?}

\item Demuestre que si $L\in NP$ entonces $L$ puede decidirse mediante un algoritmo que corre en tiempo $O(2^{n^k})$ para alguna constante $k$ (que depende de $L$).

\item Demuestre que la relación $L_1\preceq_P L_2$ es transitiva en lenguajes. Es decir que $L_1\preceq_P L_2$ y $L_2\preceq_P L_3$ implica $L_1\preceq_P L_3$.

\item Un VERTEX COVER de un grafo $G$ es u subconjunto $V'\subseteq V(G)$ tal que para todo $(u,v)\in E(G)$ tenemos que $u\in V'$ ó $v\in V'$ (o ambas). Es decir $V'$ es una colección de vértices que toca todas las aristas de $G$. El problema de decisión VERTEX-COVER consiste en decidir si dado un grafo $G$ y un entero $p$ existe un vertex cover de $G$ de tamaño $p$. En este problema estudiaremos la complejidad de VERTEX COVER:
\begin{enumerate}
\item Demuestre formalmente que VERTEX COVER esta en la clase de complejidad NP (es decir, proponga un certificado y un algoritmo de verificación y demuestre que la verificación ocurre en P).
\item Demuestre que un subconjunto $V'\subseteq V(G)$ es un CLIQUE si y sólo si $V'$ es un vertex cover del grafo complemento $\overline{G}$ (el grafo complemento $\overline{G}$ se define como el grafo con los mismos vértices que $G$ y aristas complementarias a las de $G$ (es decir, $(u,v)\in E(\overline{G})$ si y solo si $(u,v)\not\in E(G)$ y $u\neq v$).
\item Utilice la parte anterior para construir una reducción polinomial $[CLIQUE]\preceq_P [VERTEX-COVER]$.
\item Use los puntos anteriores para concluir que VERTEX COVER es un problema NP Completo.
\end{enumerate}





\end{enumerate}

\end{document}
