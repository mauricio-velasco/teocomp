
\documentclass[12pt, a4paper]{article}
\usepackage{hyperref}
\hypersetup{
  colorlinks=true,
  linkcolor=blue,
  urlcolor=cyan,
}
\urlstyle{same}
\usepackage[utf8]{inputenc}
\usepackage{amsmath}
\usepackage{amsfonts}
\usepackage{amssymb}
\usepackage{graphicx}


\newtheorem{theorem}{Teorema.}
\newtheorem{lemma}[theorem]{Lema.}
\newtheorem{corollary}[theorem]{Corolario.}
\newtheorem{definition}[theorem]{Definici\'on:}
\newtheorem{example}[theorem]{Ejemplo:}
\newtheorem{problema}[theorem]{Problema:}
\newtheorem{remark}[theorem]{Observaci\'on:}

\usepackage{graphicx}
\usepackage[spanish]{babel}
%\usetheme{default}

\newcommand{\pp}{\mathbb{P}}
\newcommand{\zz}{\mathbb{Z}}
\newcommand{\rr}{\mathbb{R}}
\newcommand{\qq}{\mathbb{Q}}
\newcommand{\RR}{\mathbb{R}}
\newcommand{\EE}{\mathbb{E}}

\usepackage{tikz, tikz-3dplot}

\definecolor{cof}{RGB}{219,144,71}
\definecolor{pur}{RGB}{186,146,162}
\definecolor{greeo}{RGB}{91,173,69}
\definecolor{greet}{RGB}{52,111,72}

\date{}

\begin{document}
\title{Pr\'actico 2 TEOCOMP: Fundamentos de probabilidad y algoritmos codiciosos.}
\author{Mauricio Velasco}
\maketitle{}


\begin{enumerate}
\item ({\it Puntos fijos en permutaciones aleatorias}) Recuerde que los puntos fijos de una permutación $\sigma\in S_n$ son aquellos índices $i\in [n]$ con $\sigma(i)=i$.
\begin{enumerate}
\item Defina la variable aleatoria $Y^{(i)}:S_n\rightarrow \RR$ con 
\[Y^{(i)}(\sigma)=
\begin{cases}
1\text{, si $\sigma(i)=i$}\\
0\text{, de lo contrario}
\end{cases}
\]
Si $\mathbb{P}$ es la medida uniforme en $S_n$, calcule $\EE[Y^{(i)}]$ dando un argumento preciso para su respuesta.
\item Use la parte $(a)$ para encontrar el número esperado de puntos fijos de una permutación aleatoria de $S_n$, elegida uniformemente.
\end{enumerate}


\item ({\it Probabilidad de éxito de construcción de permutaciones aleatorias}) Suponga que $p_1,\dots p_N$ son variables aleatorias independientes, cada una uniforme en $\{1,2,\dots, N^3\}$.
\begin{enumerate}
\item Demuestre que la probabilidad de que $(p_1,\dots,p_N)$ no tenga repeticiones es por lo menos $1-1/n$.
\item Una moneda cae cara con probabilidad $p$ y sello con probabilidad $q=1-p$. Calcule el número esperado de intentos antes de que la moneda caiga cara por primera vez.
\item Calcule el número esperado de ejecuciones que debe hacer nuestro {\bf Algoritmo de permutación uniforme mediante sorting} antes de que genere una permutación (recuerde que el algoritmo se ejecuta repetidas veces hasta que las prioridades $p_i$ salgan todas distintas asi que el problema pregunta por el número esperado de intentos antes de que esto suceda). (Sugerencia: Mezcle las partes $(a)$ y $(b)$).
\end{enumerate}

\item ({\bf Coincidencias planetarias}) El planeta bajo observación esta habitado por $k$ habitantes y dá una vuelta a su estrella cada $n$ días (un día es un giro del planeta alrededor de su propio eje).
\begin{enumerate}
\item Sea Para $i,j\in \{1,\dots, k\}$ sea 
\[X^{(ij)}=\begin{cases}
1\text{, si las personas $i$ y $j$ cumplen años el mismo día}\\
0\text{, de lo contrario.}
\end{cases}
\]
Calcule $\EE[X_{ij}]$ asumiendo que los cumpleaños se distribuyen de manera uniforme en los distintos días. Justifique su respuesta de manera precisa.
\item Sea $X$ la variable aleatoria que cuenta cuántas parejas de los $k$ individuos cumplen el mismo día. Calcule $\EE[X]$ justificando matemáticamente su respuesta.
\item Use el punto anterior para demostrar que si hay por lo menos $\sqrt{2n}+1$ individuos en un cuarto entonces deberíamos esperar que al menos dos tengan el mismo cumpleaños.
\end{enumerate}

\end{enumerate}
\end{document}




